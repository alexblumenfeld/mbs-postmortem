\documentclass[12pt]{article}
\usepackage[utf8]{inputenc}
\usepackage{amssymb}
\usepackage{amsmath}
\usepackage{textcomp}
\usepackage{graphicx}
\usepackage[a4paper, portrait, margin=1in]{geometry}
\usepackage{longtable}
\usepackage{booktabs}
\usepackage{setspace}
\usepackage[capposition=top]{floatrow}
\usepackage[style=authoryear,backend=biber,sorting=nyt,hyperref=false]{biblatex}
\addbibresource{references_30pages_onehalfspacing.bib}
\usepackage{pdflscape}
\usepackage{indentfirst}
\usepackage[implicit=false,colorlinks=true,urlcolor=blue]{hyperref}

\onehalfspacing
\begin{document}

\begin{titlepage}
	\begin{center}
		\vspace*{2cm}
		
		\large
		The Subprime Crisis Revisited: Unpacking the Long-Term Performance of a 2007 Bear Stearns Mortgage-Backed Securities Deal
		\vspace{2cm}
		
		\large
		Alex Blumenfeld\footnote[1]{Boston University and Chicago Booth; alexblumenfeld.ab@gmail.com. Funding for this project was provided by Boston University's Undergraduate Research Opportunities Program, and by Robert King of the Boston University Department of Economics.}
		
		\vspace{0.5cm}
		
		\begin{abstract}
\noindent Residential mortgage-backed securities came under intense public scrutiny after the 2007 financial crisis and the subsequent Great Recession, but have since lost their status as a significant political issue. To provide some answers about whether these securities were truly as dangerous as many suspected during the crisis, I engage in a ``postmortem'' on one securitization which was issued shortly before the financial chaos of 2007: Bear Stearns Asset-Backed Securities 2006-HE10. I find that out of an original principal pool of roughly \$1.1 billion, 44.2\% of the total principal from the deal’s mortgages has been written off due to defaults, but thanks to several provisions in the deal's structure, investors have felt only 50.6\% of these losses, giving them a principal loss rate of 23.5\% through March 2020. As for the underlying mortgages, I find that foreclosures did not peak until around 2012, showing that security principal markdowns which occurred during the financial crisis itself did not capture the full extent of the mortgages’ poor performance. I also focus on two senior tranches which were purchased by the Federal Reserve’s Maiden Lane fund in 2008, when the Fed agreed to buy seemingly toxic RMBS from Bear Stearns’s portfolio to facilitate the firm’s sale to JPMorgan Chase. One of these tranches has suffered only modest principal losses, and the other has had no principal losses whatsoever, suggesting that little in-depth analysis of these securities was conducted during the height of the crisis.
		\end{abstract}
		
		\vfill
		
	\end{center}
\end{titlepage}

\section*{Introduction}
Mortgage-backed securities (MBS) burst onto the scene in the financial markets in the 1980s and 1990s, as the promise of connecting millions of mortgage borrowers with the global capital markets offered a winning outcome for everyone involved. In theory, homeowners would get lower rates on their mortgages by gaining access to capital from all over the world, investors would obtain impressive returns from a buoyant housing market by investing in a diversified portfolio of mortgages, and banks would pocket easy revenue from originating mortgages that they could immediately move off of their balance sheets. Unfortunately, a variety of forces put pressure on the incentive structure of the securitization process, and underwriting standards for non-government-guaranteed MBS began to slip by the late 1990s and early 2000s. Whereas in the past, all mortgages included in MBS loan pools were required to conform to the standards of Freddie Mac or a similar agency, this requirement was thrown out the window in the subprime market, and the quality of the underlying collateral on many deals began to drop compared to that of past securitizations. \textcite{segoviano13} describe this as one step in a “self-reinforcing credit intermediation cycle”, where a monetary incentive to securitize as many loans as possible in order to meet demand for MBS leads to weak underwriting standards, and then the complex securitization process convinces investors that these questionable loans can be combined to form a high-quality asset, boosting the demand for these attractive securities. From here, it’s not hard to imagine the markets falling further down this slippery slope: when the U.S. housing market topped out in 2007 and subsequently crashed, the formerly bulletproof mortgage-backed securities that had been flooding the market suddenly didn’t look so attractive. After the dust settled, the banks which had engaged in the opaque process of securitization, along with the credit rating agencies that assigned had AAA ratings to securitizations full of poorly performing loans, found themselves saddled with much of the blame for allowing the financial system to reach the brink of total collapse.

Now, over a decade after the credit crunch and bank bailouts of 2007 and early 2008, one means of understanding the consequences of securitization in the wake of the Great Recession is through breaking down one specific deal that fits perfectly into the trends discussed above: Bear Stearns Asset-Backed Securities (BSABS) 2006-HE10. Consisting of just over \$1B of mortgages issued just before the financial crisis, this particular deal is a great test case for exploring how securitizations performed in the wake of the Great Recession. Among other reasons, data on its historical cash flows is easily available, its performance was clearly affected by the financial crisis, and its status as a Bear Stearns-issued deal makes it relevant to the big questions economists want to ask about mortgage-backed securities’ role in the crisis. More than a decade after the Great Recession, many important questions are still lingering, such as how drastic the crisis-induced losses on MBS actually were, whether experts’ assessments of the risk level of MBS products at the height of the crisis were correct, and whether the Fed’s decision to bail out one of the nation’s largest banks was the right move. This paper cannot provide all-encompassing answers, but focusing on a single deal will at least provide the chance to get into the weeds of how mortgage-backed securities operate and react to changing economic conditions, and the process of unpacking this deal's history will hopefully provide suggestions for how similar analysis could be done in the future.

This study is most closely related to the work of \textcite{ospina18}, who assembled a much larger MBS dataset in order to ask similar questions: they concluded that losses on AAA-rated MBS in the years after the Great Recession were relatively small, and that AAA securities were in general not as aggressively mis-rated as some suspected after the crisis. The methods I use in this paper will not be able to provide similarly broad results about the performance of residential MBS as a whole, but it has the major advantage of presenting very concrete examples of how MBS issuers thought about structuring securities that could weather financial turbulence. This strategy will allow me to evaluate the more precise question of how effectively these provisions protected investors from losses, without having to aggregate across deals with conflicting structures and rules at play.

I will also focus on the two securities from this deal which played an important role in the bailout of Bear Stearns, as one of the deal's defining features is its inclusion in the Fed’s famous Maiden Lane fund. For context, in early 2008, Bear Stearns was on the verge of running out of cash, as it depended on sources of short-term funding which were close to completely drying up. In order to avoid a collapse which would cause credit markets to seize up even further, the Fed chose to facilitate the purchase of Bear’s assets by JPMorgan Chase. JPMorgan claimed that they wouldn’t be able to completely take over the remnants of Bear Stearns if they were forced to acquire the toxic MBS tranches that were still on Bear’s books, so the Fed Board of Governors authorized a \$29 billion loan to a brand-new New York Fed vehicle called Maiden Lane LLC, which would have the job of taking on the riskiest securities from what remained of Bear Stearns’s balance sheet \parencite{fcic09}. Several varieties of MBS were included, and as it turns out, two entire tranches from BSABS 2006-HE10 made their way into the Fed’s hands as part of the Maiden Lane agreement: I-A-3 (CUSIP: 07389RAC0) and II-1A-3 (CUSIP: 07389RAQ9), with initial principal balances of \$11,213,000 and \$20,339,000, respectively \parencite{fcic09}. Because these securities demonstrate the extent to which the government had to intervene in the financial system to prevent a broader collapse in 2008, it’s worth asking whether they actually caused the government to lose any money, or whether they weren’t really as toxic as JPMorgan Chase’s analysts believed during the chaos of March 2008.

In the sections that follow, I use the data from these securities' monthly investor reports to provide several perspectives on the deal’s history. Part I briefly covers background information on MBS terminology and structures, and it lays out key information about this deal’s structure. Part II provides a high-level overview of the deal’s performance and some context on important events throughout its lifetime. In Part III, I look at the deal’s mortgage pool directly, in order to learn more about the payment and delinquency patterns that would eventually result in losses for investors. Finally, in Part IV, I bring the focus back to the actual securities, by investigating to what extent investors throughout the subordination hierarchy were protected from losses on the deal’s underlying mortgages, with extra attention given to the two Maiden Lane-owned tranches. The conclusion offers further commentary on how this deal’s tumultuous life fits into the larger conversation about mortgage-backed securities and their role in the financial system.

\section*{Part I: Background and Deal Details}
\subsection*{Background}
Before jumping into the specifics of this deal, I will briefly review the methods that MBS issuers use to insure investors against losses and make their securities appear less risky, because understanding this jargon is critical for any serious analysis of an MBS deal. In the diverse world of “non-agency” (non-government-sponsored) MBS, issuers have come up with a wide array of strategies for protecting investors’ capital and limiting the impact of mortgage defaults on securities’ returns, referred to as ``credit enhancement''. The simplest type of credit enhancement is the division of a deal into several tranches (separate securities), which take on losses in a specific order. For example, with an initial pool of \$100 million of mortgages to securitize, an issuer could offer a low-yielding “senior” tranche of \$70 million and a higher-yielding “mezzanine” tranche of \$30 million, with the stipulation that mortgage defaults will be applied to the mezzanine tranche as principal writeoffs before being applied to the senior tranche. This hierarchy offers higher returns to investors in mezzanine tranches – the ones whose balances are marked down first in the event of mortgage defaults – in exchange for taking on default risk. In addition, when borrowers make their principal payments, that cash is generally used to pay off the highest-ranking tranche’s principal balance first, so senior-level borrowers can be confident that they’ll be the first to get their money back.
	
On top of a simple subordination structure, two more sophisticated mechanisms are common in the MBS world, and the deal I examine here utilizes both of them. The first, called excess spread, or XS for short, refers to the practice of using cash from leftover interest payments from the mortgages to mitigate the effects of principal losses. When borrowers default and the trustee is about to be forced to write down the principal balance on the lowest-ranked tranche, the trustee will first check on whether there’s any cash left over from borrowers’ interest payments that hasn’t been distributed to the securities as interest; this could happen if the interest rates that borrowers pay are higher than the rates that the securities offer to investors. If this is the case, the trustee can use that extra cash to pay off a security’s principal balance directly, instead of having to permanently record a loss by marking down the security’s balance. In good times, if there are no losses to be mitigated with the available excess spread, the proceeds from this extra interest will instead flow to a specialized tranche within the deal, which investors can purchase as a way of betting that the deal will perform so well that there will be lots of leftover interest payments to be captured.
	
The second more complicated credit enhancement mechanism is effectively a modified first-loss tranche, referred to as an overcollateralization buffer, or just OC. To increase the volume of defaults that can be tolerated before investors have to suffer losses, a common practice in securitization is to offer securities whose total balance is less than the total balance of the underlying loans. For instance, a deal consisting of mortgages with a collective principal balance of \$100 million might offer only \$90 million of securities to interested investors, with a \$10 million “OC buffer” acting as the first-loss tranche and retained by the bank issuing the securities. This way, investors are protected from the first \$10 million of mortgage defaults, because those losses are charged to the OC buffer and not the lowest-ranked mezzanine tranche. The OC tranche might receive interest payments depending on the deal’s structure, and if the mortgages perform well, the “overcollateralization ratio” of the deal could increase, meaning that the senior and mezzanine securities are having principal paid off while the level of the OC buffer stays steady. Many deals have provisions where the allocation of cash to investors changes if the OC ratio rises above a certain threshold, but in the case of the deal we’ll examine, which started taking on heavy losses shortly after being issued, these provisions were never triggered.
	
	The final tool that Bear Stearns used to address uncertainty in this deal's cash flow patterns was a so-called swap agreement. In an MBS deal with mortgages that pay fixed rates and securities that pay floating rates, the issuer needs insurance against a rise in interest rates, which would cause them to have to pay out more interest to investors without receiving any more cash from borrowers. So, they can set up a simple swap contract with a counterparty who prefers to receive a fixed rate: the two parties settle on a particular fixed baseline rate, a floating rate to track, and a total principal balance that determines the magnitude of payments. Then, at the end of each month, if the current floating rate is above the stipulated fixed rate, the counterparty sends money to the issuer, and if the floating rate ends the month below the agreed-upon fixed rate, the MBS issuer has to send a payment to the counterparty. Swap contracts, subordination, and XS/OC aren’t the only elaborate MBS structuring techniques that have been devised over the years, but they’re all important for understanding the securities I look at here.


\subsection*{Deal Details}

The entire securitization market is so large and diverse that it would be impossible to give an exhaustive analysis of how every single combination of credit enhancement structures and underwriting standards held up during the financial crisis, so I have focused on this single deal in order to provide an in-depth analysis of its structure and performance.

BSABS 2006-HE10 was built from two totally disjoint mortgage pools, called Group I and Group II. At the deal’s inception in late 2006, Group I consisted of about \$320 million of mortgages, which came with no guarantees about whether they conformed to Freddie Mac or Fannie Mae standards, and it corresponded to its own hierarchy of securities: from top to bottom, there were the senior tranches I-A-1, I-A-2 and I-A-3, and then mezzanine tranches I-M-1 through I-M-9. Group I also included several other non-marketed securities: I-R-1, I-R-2, I-R-3 and I-RX held the “residual interest” (leftover excess spread) on the deal, while I-CE represented the OC buffer and I-P collected prepayment penalties. Group II was made up of about \$830 million in mortgage principal, divided among three subgroups: the \$391 million of loans in subgroup II-1 didn’t necessarily conform to any federal standards, while the \$265 million balance in subgroup II-2 was completely Fannie Mae-conforming and the \$174 million balance in subgroup II-3 was completely Freddie Mac-conforming. Group II’s loans backed five senior securities – II-1A-1, II-1A-2, II-1A-3, II-2A, and II-3A, all of which would take losses on a pro rata basis if all mezzanine securities were wiped out – along with nine mezzanine securities, labeled II-M-1 through II-M-9. Group II also had the same types of non-marketed auxiliary securities as Group I: II-R-1 and II-RX for the residual interest, II-CE for the OC buffer, and II-P for prepayment penalties.

The set of marketed securities (all “A” and “M” tranches) offered various floating rates tied to one-month USD LIBOR, from LIBOR + 11 bps for I-A-1 to LIBOR + 250bps for I-M-9. In the mortgage pool itself, there were both fixed- and adjustable-rate loans, with the adjustable-rate borrowers paying rates which varied with six-month LIBOR. Just like many of the other mortgages being issued at the time, all of the adjustable-rate mortgages in this deal came with initial fixed-rate “teaser” periods of two, three or five years, which would seem to imply a certain degree of stability in payments for at least the period of 2007-2008, before borrowers felt the full effects of adjustable rates.

In addition to the standard subordination hierarchy, each group of securities had an excess spread-overcollateralization (XS/OC) credit enhancement structure in place. The two groups were also set up with their own five-year swap agreements based on a principal balance that would shrink from 2007 until the contract’s end in 2011, which required the MBS trust to pay out a fixed rate of 4.9350\% (for Group I) or 4.9315\% (for Group II) each month to the counterparty, Wachovia, in exchange for a variable rate tied to one-month LIBOR.

The loans which comprised the mortgage pool for this deal were originated mainly by Encore Credit Corporation, a national subprime lender which would be bought by Bear Stearns in February 2007, as well as EMC Mortgage Corporation, a direct subsidiary of Bear Stearns at the time \parencite{cpi09}.  After Bear Stearns bought all of the deal’s component mortgages from their originators, servicing duties were fully passed off to EMC. Unfortunately, not too long after BSABS 2006-HE10 was first marketed to investors, EMC and Bear Stearns would end up being jointly fined \$28 million by the FTC for “unlawful and abusive collection practices” as well as “paying [sic] inadequate attention to the integrity of consumers’ loan information” \parencite{ftc08}. This deal also became involved in a larging lawsuit alleging that Bear Stearns and other banks had misled investors as to the quality of the loans that made up the MBS products they were offering, and this concluded in 2013 with a massive payout for the investors; these legal proceedings will be discussed further in the Conclusion \parencite{doj13}. Finally, the trustee which oversaw this deal’s operations and produced monthly remittance reports for investors was LaSalle Bank, although USBank eventually took over these duties in mid-2013, leading to some changes in the types of information provided in the investor reports which made parts of my analysis more difficult.

\section*{Part II: Timeline of Important Events and Summary of Securities' Performance}

\subsection*{Timeline}

This deal has persisted through chaotic times, being issued at the height of the early-2000s home price bubble, then suffering during the ensuing financial crisis, and then being faced with a lawsuit over misrepresentation of borrowers’ ability to pay off their mortgages. For context, Figure \ref{fig:timeline_of_important_events} provides a timeline of key events in the deal’s history:

\begin{figure}[h]
	\centering
	\caption{Timeline of Important Events for BSABS 2006-HE10}
	\includegraphics[width=0.8\textwidth]{../figures/timeline_of_important_events}
	\label{fig:timeline_of_important_events}
	\floatfoot*{Notes: This deal was originated in early 2007, faced its first credit rating downgrades later that year, and started having principal written off due to defaults on the underlying mortgages in 2009.}
\end{figure}

\subsection*{Overall Levels of Writedowns}

Over a decade after their initial issuance, I will begin by attempting to get a high-level picture of how much each security has suffered in principal writedowns. In Group I, all of the mezzanine securities from I-M-3 through I-M-9 have had 100\% of their principal written off due to poor performance by the mortgage pool, tranche I-M-2 has had approximately 78\% of its principal written off, and tranche I-M-1 has emerged from the financial crisis unscathed, taking no writedowns, but not receiving any principal payments either. In Group II, realized losses have made impacts even higher up the subordination totem pole, with every single mezzanine tranche being completely wiped out -- each of the Group II senior tranches are now taking losses, which are allocated pro rata within that collection of securities. Overall, mezzanine securities have had 91\% of their principal written off, while senior securities across the two groups have lost only about 5\% of their original face value through writedowns. Of an original total marketed certificate balance of \$1.09 billion, only \$182 million is still outstanding, and \$258 million of the certificate holders’ original investment has been lost due to defaults among the deal’s underlying mortgages, for a loss rate of 23.5\%, a figure that would surely make most investors cringe. So, in a vacuum, before considering the actual principal and interest payments that have been made to these securities throughout their lives, we can say that 23\% of the original face-value principal balance in this deal will never be paid back to investors. Of course, because of the credit enhancement systems built into MBS deals, these losses are allocated far from evenly, so understanding how any individual security performed requires more work.

To get a rough estimate of the degree of credit protection experienced by a particular security at any point in time, I made graphs which represent the composition of the remaining principal in each group of securities, split up across the various tranches. So, everything below a particular security’s band on the graph is “protecting” it, in the sense that mortgage losses equal to the total remaining balance on all of the more subordinated securities would have to occur before the one in question would suffer any writedowns. One other helpful feature of this type of graph is that it distinguishes between principal payments and writedowns in terms of how they affect the deal’s overall principal balance. Securities that have their principal balances fully paid off by remittances from mortgage borrowers slowly disappear at the top of the graph, and securities which are fully written off - due to the servicer confirming losses on mortgages - disappear at the bottom of the graph.


\begin{figure}[h]
	\centering
	\caption{Subordination Levels in Group I}
	\includegraphics[width=0.7\textwidth]{../figures/stackplot_share_of_principal_group_i}
	\label{fig:stackplot_share_of_principal_group_i}
	\floatfoot*{Notes: This chart shows the distribution of Group I's principal across its tranches at each point in time. Mezzanine securities that have their principal marked down to zero disappear off the bottom of the axis, while senior securities that are fully paid off disappear at the top.}
\end{figure}

\begin{figure}[h]
	\centering
	\caption{Subordination Levels in Group II}
	\includegraphics[width=0.7\textwidth]{../figures/stackplot_share_of_principal_group_ii}
	\label{fig:stackplot_share_of_principal_group_ii}
	\floatfoot*{Notes: This chart shows the same information as Figure \ref{fig:stackplot_share_of_principal_group_i}, but for the Group II securities.
	
	Data Note: After all mezzanine tranches in Group II were written off, further losses were allocated to all senior tranches simultaneously.}
\end{figure}

Figures \ref{fig:stackplot_share_of_principal_group_i} and \ref{fig:stackplot_share_of_principal_group_ii} provide a useful framework for understanding how confident a holder of a senior-level security can feel about eventually getting their money back. In Group I, at least 20\% of the remaining principal balance is always taken up by mezzanine securities, so at every point in time thus far, senior-level investors have known that over 20\% of the remaining security principal would have to be written off before they would be hurt at all. However, in Group II, by mid-2014, the senior securities were directly exposed to any future realized losses on the mortgages, since the buffer provided by the existence of mezzanine securities had completely evaporated. These figures also illustrate one easily-overlooked point about the nature of mortgage-backed securities. Despite the popularity of stories of daring hedge fund managers like Michael Burry winning big by betting against risky mezzanine tranches, the majority of the principal is concentrated in the senior securities, which offer relatively low interest rates and will take many years to be fully paid off.


\subsection*{Importance of ``Lower Senior'' Tranches}

The subordination-levels graphs also illustrate one way of interpreting desirability of each kind of security in this deal, and in MBS deals in general. The highest-level senior securities (like I-A-1 in Group 1) have very low interest rates, but are also low-risk and are likely to be paid off first, making them a great choice for institutions that need to park their cash somewhere with a safe positive return. The mezzanine tranches are significantly higher-risk, but they present much more attractive interest rates, for the exact reasons that this deal demonstrates – they will likely be the last to receive principal payments, and they do not enjoy the benefits of subordination to the extent that senior securities do. This leaves the “lower senior” securities, like I-A-3 in Group I and II-3A in Group II, in an odd state of limbo. Because they are still senior-quality and are unlikely to suffer significant losses, they don’t offer very attractive rates, and yet they are also likely to be the last securities still standing in the deal. As shown in Figure 2, these lower senior tranches make up the majority of the principal balances remaining in both groups at the moment, since the highest-ranked senior securities (at least in Group I) have been fully paid off, while the mezzanine securities have been mostly wiped out by losses. Due to the mortgages’ overall poor performance, the lower senior securities are still receiving only meager principal and interest payments, so investors in these particular tranches will likely have their money tied up in this deal for quite a while.

In my opinion, this makes these securities somewhat less desirable in comparison to the ``higher senior'' and mezzanine securities, as “lower senior” status offer neither the near-absolute insulation from principal losses of a I-A-1 nor the chance at excellent yields offered by those such as a I-M-7 or a I-M-9. So, beyond the simple ordering of securities presented by a deal’s prospectus, in which each tranche is simply less risky than all those below it, breaking down these securities into three general categories may be a helpful tool for understanding deals like this one from the perspective of investors. Despite the fact that all of the securities in Group I are based on the same underlying mortgage pool (and likewise for Group II with its own mortgage pool), the above figures suggest that investors should have significantly different expectations for how their money will perform depending on where they choose to locate themselves within this hierarchy of securities.


\section*{Part III: The Mortgages}
\subsection*{Summary of Cash Flows}

I will now work towards a bottom-up understanding of the deal's performance since 2007 by examining its mortgage pool in greater detail. From this point onward, even though this deal’s securities are technically based on two separate groups of mortgages, Group I and Group II, the two groups will be referred to jointly when discussing their performance. Table \ref{tab:table_deal_summary} contains some useful information about the mortgages’ performance.

\begin{table}[h]
	\centering
	\caption{Mortgage Performance Summary}
	\begin{tabular}{| l  | l  |}
\hline
Mortgage pool size in January 2007 & \$1.143 billion \\
Mortgage pool size in March 2020 & \$183 million \\
Total prepayments on mortgages & \$294 million \\
Total principal remitted to the MBS trust & \$461 million \\
Total realized losses on the mortgage pool & \$510 million \\
Mortgage loss rate (realized losses divided by original pool size) & 44.6\% \\
\hline
\end{tabular}
	
	\label{tab:table_deal_summary}
\end{table}

Even just this high-level information is enough to tell a story about these mortgages’ performance: realized losses on this group of mortgages, which covered wide ranges of borrower creditworthiness, principal size, and geographic location, were over 40\% of the initial principal balance, a figure that would cause any sensible investor to shudder. This alone makes it clear that the quality of these mortgages was shaky at best. Cumulative realized losses have also exceeded the total principal payments distributed to investors since January 2007, leading to the remarkable conclusion that of all the mortgage principal that has dropped out of the deal since its inception, roughly half has been “bad” rather than “good”.

\begin{figure}[h]
	\centering
	\caption{Cross-Section of Loan Delinquency Statuses in Each Month}
	\includegraphics[width=0.8\textwidth]{../figures/stackplot_delinq_status_with_writeoffs}
	\label{fig:stackplot_delinq_status_with_writeoffs}
		\floatfoot*{Notes: This figure shows the distribution of loan statuses for all mortgages included in the deal, ranging from Current to Foreclosure/REO. It also includes the destination of every dollar of principal that has eliminated from the deal, through either payments to investors or writedowns. The number of loans in foreclosure jumped up after the Great Recession and remained relatively high until 2014. In addition, of all principal that has ``exited'' the deal, more has been written off due to mortgage defaults than has been paid out to investors.}
\end{figure}

Figure \ref{fig:stackplot_delinq_status_with_writeoffs} shows that the turmoil in housing markets in the wake of the Great Recession had effects on this deal until roughly 2014, far beyond the end of the financial crisis. Specifically, as the Great Recession reached its peak, there was a major uptick in the total value of loans for which borrowers were at least 3 months delinquent, and it took until around mid-2010 for this to fall back down. Of course, delinquency is only the beginning of the story: after the loans that are severely delinquent move through the bankruptcy process, they end up in foreclosure, where Bear Stearns would try to recoup as much of its losses as possible by selling off the borrower’s house. Once the underlying home is foreclosed on, investors become exposed to the demand for foreclosed-upon homes, not just the financials of the original borrowers. Looking at only the total loan balances held by delinquent borrowers would suggest that the detrimental effects of the Great Recession had mostly worn off by 2011 or so, but also looking at foreclosure and bankruptcy rates gives a more complete picture of how the housing market crash affected this deal. As shown by the brown region of Figure \ref{fig:stackplot_delinq_status_with_writeoffs}, it took until 2014 for the glut of foreclosed-upon loans in the deal to mostly dissipate.

	As the total remaining mortgage principal balance shrinks, the rest has to go somewhere, and that principal can be sorted in two buckets: “remittances to trust” – including prepayments and regularly scheduled payments – and “written off”. Investors receive the former, and the latter is defaulted-on principal that cannot be recovered. These two components are included at the top of Figure \ref{fig:stackplot_delinq_status_with_writeoffs}. The “Cumulative Non-Prepaid Principal to Trust” segment is calculated as total principal remitted to the trust minus total mortgage prepayments, the “Cumulative Prepayments” segment is the total value of prepayments by borrowers, and the “Cumulative Principal Written Off” segment is the sum of all realized losses on the mortgage pool over time. For an investor, it would be quite disconcerting to see that over \$500 million of the original principal balance in the mortgage pool has been written off due to defaults (as of March 2020), while slightly less than that has actually been paid out to security holders after 12 full years of mortgage payments. Part IV will provide more detail on how investors were affected by these losses.

\subsection*{Understanding the Deal's Prepayment Patterns}

In addition to the realized losses that are so important for determining the value of investors’ MBS holdings, prepayments introduce an extra layer of uncertainty, and the evolution of the prepayments for this deal’s mortgages gives another perspective on how the financial crisis profoundly affected mortgage borrowers. Prepayments through March 2020 totaled almost \$300 million, larger than the sum of regularly scheduled principal payments, but they were also very unevenly distributed through time, as shown in Figure \ref{fig:timeseries_prepayments}.

\begin{figure}[h]
	\centering
	\caption{Time Series of Mortgage Prepayments}
	\includegraphics[width=0.8\textwidth]{../figures/timeseries_prepayments}
	\label{fig:timeseries_prepayments}
	\floatfoot*{Notes: This figure shows the time series of prepayments on the deal's mortgages. This deal saw a rapid drop in prepayments shortly after being issued in January 2007, in contrast to typical MBS deals that have high prepayment levels for 3-4 years after origination.}
\end{figure}

In normal times, a high level of prepayments can spark enthusiasm for investors in mezzanine tranches, as it can be a strong indicator that a “stepdown” is coming. This means that the deal is performing so well that instead of making principal payments to the most senior tranche first, which is the usual structure, the cash flow hierarchy will change so that the lowest tranche will get principal payments each month instead. This practice of trying to pay off the lowest-rated securities first has the potential to save a lot of money for the issuer: the lowest tranches pay out the highest interest rates of any securities in the deal, and in some MBS designs, the issuer will get to pocket any extra interest that’s left over after making all required payments to the remaining securities. If this stepdown occurs, speculators in high-risk tranches will have reason to become much more confident in their future returns. On the opposite end of the spectrum, though, a trend of increasing delinquencies and foreclosures rather than prepayments would spell danger even for securities with substantial credit enhancement, and the rapid decline in prepayments seen in Figure \ref{fig:timeseries_prepayments} seems to reflect a drastic change in borrowers’ behavior.

	There is a substantial literature on the causes of mortgage prepayments, but one concise (if very early) summary comes from a review article by Sean Becketti, an economist at the Kansas City Fed, who lists three main events that can trigger rapid changes in a mortgage’s principal balance: refinancing, relocation, and default \parencite{becketti89}. He points out that the most important driver of refinancing is the relative gap in interest rates between a borrower’s current mortgage and others available on the market, while relocation is more intuitively driven by borrowers’ major life events (like changing jobs or getting married), and default is strongly related to general macroeconomic conditions. Using data on government-backed Fannie Mae MBS, Becketti reached the conclusion that mortgage borrowers’ behavior follows a clear pattern: prepayments are low at first, then they peak three or four years after origination due to relocations and the desire to refinance, and then they settle down to a stable long-term level after that.
	
If this really is the standard pattern of prepayments, then BSABS 2006-HE10 defies the usual trends seen in MBS loan pools, as it had tens of millions of dollars in prepayments within its first 1.5 years of existence, but then prepayments dropped to near-zero levels as the Great Recession took hold.  In this case, it appears that any appeal of refinancing driven by the Fed's decision to slash interest rates was overwhelmed by the economic distress that borrowers felt due to the national recession. So, when such a massive macroeconomic shock occurs, it seems to become more difficult to differentiate between forces driving prepayments and forces driving defaults. If there really were a large number of borrowers in this deal who were financially comfortable and rationally examined interest rate patterns to decide when to refinance, then there might have been a more standard prepayment time path, where prepayment rates remain for three or four years until the deal is “seasoned” (the industry term for a deal whose prepayment risk has reached a stable level). Instead, while there was a slight rebound in prepayments in 2008 as the Fed cut rates, delinquencies spiked while prepayments mostly plummeted, suggesting that the state of the economy was a much more immediate concern for borrowers than the option to potentially shave a few basis points off their interest rates by putting time and money into refinancing.

In the big picture, both prepayments and delinquencies reached very low levels by the second half of the 2010s, so most borrowers at this point are making their payments each month but not doing much else. Although experts did acknowledge that MBS deals exhibit “considerable idiosyncratic prepayment risk” \parencite{becketti89} decades ago, there is still much more to be done in terms of understanding how both financial and macroeconomic factors influence a single borrower’s preferences – they’re always thinking about refinancing, defaulting, relocating, or doing none of the above – rather than considering each of these possible behaviors in isolation.

\subsection*{Fixed vs. Adjustable-Rate Loans}

One other useful angle from which to analyze the composition of this deal’s mortgage pool is by breaking it down into fixed- and adjustable-rate loans. A major part of the public narrative about the collapse of the housing market in 2007 was the prevalence of adjustable-rate loans issued with weak underwriting standards to households with lackluster credit scores, so it makes sense to take a look at how risky this deal’s adjustable-rate loans actually turned out to be, compared to their fixed-rate counterparts.

In early 2007, adjustable-rate mortgages (ARMs) made up roughly 76\% of the deal's total mortgage pool (as shown in Table \ref{tab:table_realized_losses_fixed_vs_arm}), matching up with the common perception that banks specifically chose to focus on originating adjustable-rate loans – which would potentially offer higher returns for investors if global rates like LIBOR remained high – in order to securitize them. At this point, it should be mentioned that this “baseline” composition of the pool here is actually from the March 2007 data, not when reports were first issued in January 2007, since the number of loans in the deal doesn’t actually peak until March; my best guess as to why this occurs is that the loans in the deal were all originated in 2006, but it took until March of 2007 to finish packaging all of them into this securitization. With that in mind, the first plot of interest when comparing fixed vs. ARM loans is a comparison of overall delinquency rates between the two groups, shown in Figure \ref{fig:timeseries_delinq_stats_fixed_vs_arm}.

\begin{figure}[h]
	\centering
	\caption{Delinquency Rates for Fixed- and Adjustable-Rate Mortgages}
	\includegraphics[width=0.8\textwidth]{../figures/timeseries_delinq_status_fixed_vs_arm}
	\label{fig:timeseries_delinq_stats_fixed_vs_arm}
	\floatfoot*{Notes: The left panel shows the share of remaining loans in the deal whose borrowers were up-to-date on payments, and the right panel shows the share of remaining loans in bankruptcy or foreclosure. Adjustable-rate loans consistently had higher bankruptcy and foreclosure rates than fixed-rate loans.}
\end{figure}

The left plot of Figure \ref{fig:timeseries_delinq_stats_fixed_vs_arm} displays the portion of each month's remaining loan balanace for which borrowers were still up to date on their payments. For example, it shows that out of all ARM borrowers whose loans were still outstanding in early 2010, less than 35\% of them were still staying current on their mortgage payments. The right-hand plot shows the population of the other end of the delinquency spectrum, specifically what percentage of the remaining loan balance was in either bankruptcy or foreclosure. Taken together, these two plots make it clear that borrowers with adjustable-rate loans were hit harder by the Great Recession than their fixed-rate counterparts, as ARM borrowers were more likely to enter bankruptcy and less likely to stay current on payments than their fixed-rate counterparts. In order to see how this differential might translate to losses on investors’ securities, the next step is to check on what the cumulative “loss rates” were, i.e. what percentage of the original principal balance had been written off as realized losses after the bankruptcy/foreclosure process was complete. Unfortunately, due to a change in trustee for this deal in 2013, the investor reports after May of that year do not break out the monthly realized loss figures into adjustable- and fixed-rate segments, so it is only possible to investigate this component of the deal up through mid-2013. This summary information for the aggregated fixed- and adjustable-rate loan groups is displayed in Table \ref{tab:table_realized_losses_fixed_vs_arm}. In addition, Figure \ref{fig:timeseries_cumulative_losses_fixed_vs_arm} gives some extra context to these figures by plotting the cumulative loss rate (the third column in the table) over time.

\begin{table}[h]
	\centering
	\begin{tabular}{| l | l | l |}
\hline
  & \textbf{ARM} & \textbf{Fixed} \\
\hline
Number of Loans (March 2007) & 3,405 & 1,668 \\
Principal Balance (March 2007) & \$847 million & \$268 million \\
Loss Rate (baseline is March 2007) & 32.0\% & 30.0\% \\
Share of Initial Principal & 76.0\% & 24.0\% \\
Share of Realized Losses Through May 2013 & 77.3\% & 22.7\% \\
\hline
\end{tabular}
	\caption{Realized Losses by Mortgage Type (through May 2013)}
	\label{tab:table_realized_losses_fixed_vs_arm}
\end{table}

\begin{figure}[h]
	\centering
	\caption{Cumulative Principal Losses for Fixed- and Adjustable-Rate Mortgages}
	\includegraphics[width=0.7\textwidth]{../figures/timeseries_cumulative_losses_fixed_vs_arm}
	\label{fig:timeseries_cumulative_losses_fixed_vs_arm}
	\floatfoot*{Notes: This figure shows the cumulative path of mortgage default rates for fixed- and adjustable-rate mortgages. ARMs have slightly higher loss rates in terms of total principal written off, but this is difficult to assess fully because this breakdown of the data is not available after 2013.}
\end{figure}

So, contrary to what the previous graphs suggested about ARM loans performing much worse than fixed-rate loans, the available data suggest that adjustable-rate loans were not all that much more toxic than fixed-rate loans – by May 2013, adjustable-rate mortgages had had 32\% of their March 2007 principal balance wiped out through write-offs, while the corresponding figure was 30\% for fixed-rate mortgages. However, this doesn't necessarily indicate that ARM loans are actually no riskier for investors than their fixed-rate counterparts. My preferred explanation for why the stories told by Figure 6 and Table II are so different is that not all of the realized losses which resulted from the delinquencies shown in Figure 6 were actually finalized and booked by May 2013. As far as I can tell after reading through this deal’s documentation, realized losses on each delinquent loan are only recorded after a loan has completed the foreclosure process and the servicer has liquidated the house for as much as they can get. So, because Figure \ref{fig:stackplot_delinq_status_with_writeoffs} tells us that there were still a significant number of loans in the foreclosure stage in May 2013, it's plausible that even though realized losses on those loans were in fact disproportionately weighted towards ARMs, and that the associated losses were not booked until after mid-2013.

In summary, the data available for this deal do suggest that adjustable-rate loans performed somewhat worse than their fixed-rate counterparts in terms of borrowers staying current and avoiding default, but I cannot fully quantify how this disparity affected MBS investors due to limitations with the mortgage principal loss series in the investor reports. Now, I will circle back to the securities that are the true focus of  this analysis and hunt for some more answers about just how toxic they turned out to be.

\section*{Part IV: Consequences for MBS Investors}

\subsection*{Where Are the Securities' Principal Payments Coming From?}

The fundamental appeal of an MBS, or any similar type of asset-backed security, is that the security’s complex structure softens the blow felt by investors when the underlying asset performs worse than expected, so the big question at hand is how much benefit that complex structure provided. This is partially about determining what fraction of losses on mortgages ended up affecting the overall group of security holders, but it’s also important to know how holders of different tranches were affected. The starting point for this analysis is Figure \ref{fig:timeseries_losses_vs_writedowns}.

\begin{figure}[h]
	\centering
	\caption{Principal Writedowns on Mortgage Pool and Securities}
	\includegraphics[width=0.8\textwidth]{../figures/timeseries_losses_vs_writedowns}
	\label{fig:timeseries_losses_vs_writedowns}
	\floatfoot*{Notes: This figure compares total principal markdowns on the securities (blue) to total defaults in the mortgage portfolio (orange). Even after the deal's overcollateralization buffer ran out, there continued to be a wedge between mortgage losses and security losses, due to the presence of excess spread coming from fixed-rate mortgages.}
\end{figure}

At a very high level, this is par for the course for a poorly performing MBS deal: the overcollateralization buffer is rapidly used up in an effort to cover mortgage losses, and writedowns on the MBS’s constituent tranches rise immediately afterwards. Perhaps surprisingly, after the initial OC buffer goes to zero, there’s still a significant gap between mortgage losses from defaulting borrowers and principal losses on securities that directly affect investors. The most likely explanations for this are excess spread due to consistent interest payments and extra cash generated by the deal’s swap agreement, so the next step is to figure out which effect dominated.

	Because the prospectus and other documents for BSABS 2006-HE10 are incredibly long and dense, it turns out that even a simple-sounding task such as trying to figure out which income streams compose the principal payments that are sent to security holders each month can be quite difficult. The data in the investor reports ultimately showed that the main sources of the cash that was sent out to investors as principal payments were 1) principal payments from the mortgages themselves, 2) interest payments from the mortgages exceeding those which had to be made to the securities, i.e. excess spread, and 3) cash coming in from the swap provider, which was negative on net. Although it’s not a perfect fit, the sum of these three components still matches up well with the series of certificate principal payments, as shown in Figure \ref{fig:timeseries_security_principal_pmts_composition}, confirming that these are the main factors at play.

\begin{figure}[h]
	\centering
	\caption{Main Components of Principal Payments Received by Securities}
	\includegraphics[width=0.7\textwidth]{../figures/timeseries_security_principal_pmts_composition}
	\label{fig:timeseries_security_principal_pmts_composition}
	\floatfoot*{Notes: This figure compares the total principal payments made to the securities to the sum of three components: principal paid by mortgage borrowers, excess spread generated by borrowers' interest payments, and the net payment to the MBS trust from the swap agreement counterparty. The sum of these three cash flows mostly matches up with the total principal paid out to investors.}
\end{figure}

To calculate the true extent of excess spread created throughout this deal’s life, I need both 1) the total interest remitted from the mortgage servicer to the trust and 2) the total interest paid out to the securities each month. It turns out that from January 2007 to March 2020, the mortgages in this deal generated \$249 million of excess spread. As described in Part I, when mortgage writedowns rise, the trustee is forced to either write down an MBS tranche or use excess spread (if any is present) to pay off some security's principal. If there had been no excess spread, then in the event that the servicer had to write off some mortgage principal due to defaults, the trustee would have had no choice but to write down the lowest-rated security remaining in the deal by the full amount of the mortgage writeoffs. However, when the trustee has some extra cash available each month due to excess spread, they can make extra principal payments to the top-rated remaining security, thus reducing the magnitude of the monthly markdowns on the bottom tranche. In this deal, the sequence of events each month appears to be as follows:

\begin{enumerate}
	\item Borrowers make their mortgage payments
	\item The resulting principal and interest are sent to the trust
	\item Interest payments are made to securities
	\item Excess spread is calculated
	\item Excess spread is distributed as principal payments to the top MBS tranche
	\item The lowest MBS tranche is written down by the necessary amount to make up the difference between mortgage pool shrinkage and security balance shrinkage
\end{enumerate}

In the long run, since some of the top-level securities’ principal was being directly paid off using any available cash from excess spread, only a fraction of the principal losses on the mortgages actually flowed through to the securities as writedowns. This is one feature of the deal’s history that was specifically made possible by a low-interest-rate environment: while all of the securities paid out a floating interest rate to investors, 24\% of the mortgages were fixed-rate, so when rates dropped significantly in response to the Great Recession, the fixed-rate borrowers ended up providing more interest than what needed to be paid out on the adjustable-rate securities. While this discussion of security vs. mortgage principal shrinkage may seem like nothing more than an accounting puzzle, the result of it is that we now have a clearer understanding of how important the deal’s excess spread provision was for mitigating realized losses, and we can see that it played a much larger role than the other available credit enhancement mechanism, the initial overcollateralization buffer.


\subsection*{The Swap Agreement}

In addition, at the level of the entire deal, not a single security, the final factor that influenced the extent to which investors felt the effects of poor mortgage performance was the deal’s swap agreement. It’s technically two separate agreements (one for Group I, one for Group II), but the main idea is that from 2007 through 2011, the trust (the “Swap Administrator”) received floating-rate payments from Wachovia (the “Swap Provider”) while paying a fixed rate back to Wachovia. Each month, the trust paid out a rate of 4.935\% (for Group I) or 4.9315\% (for Group II) on a continually declining balance, which was the monthly “notional value” of the swap, and Wachovia sent back the one-month LIBOR rate. The overall performance of the two swap agreements is shown in Figure \ref{fig:timeseries_swap_performance}.

\begin{figure}[h]
	\centering
	\caption{Performance of Swap Agreement}
	\includegraphics[width=0.7\textwidth]{../figures/timeseries_swap_performance}
	\label{fig:timeseries_swap_performance}
	\floatfoot*{Notes: The top panel shows the monthly cash flow from the swap counterparty to the MBS trust, and the bottom panel shows the time series of one-month LIBOR. After central banks slashed interest rates in 2008, the MBS trust saw net cash outflows go to the swap counterparty until the end of the contract in 2011.
	
	Data Note: LIBOR data comes from the Federal Reserve Bank of St. Louis}
\end{figure}

While getting exposure to a floating-rate instrument may have seemed like a great idea in 2006 when the deal’s mortgages were being underwritten, the swap agreement turned out to be a net negative for this collection of securities. This can be seen in Figure \ref{fig:timeseries_swap_performance}, as the trust overseen by LaSalle (and then by USBank) ended up paying out millions more than it received over the life of the swap provision. This figure clarifies the main takeaway from the deal's swap agreement, which is that receiving a floating rate and paying out a fixed rate is only beneficial if interest rates stay high. As the financial crisis worsened and central banks began cutting rates throughout 2008, LIBOR fell dramatically. Before this, the MBS trust had been profiting from receiving that floating rate, but from 2008 onward, it ended up having to use some of its monthly mortgage proceeds to pay the swap provider. On net, the trust ended up paying out approximately \$39 million to the swap provider during the swap agreement’s five-year life. Of course, the main channel through which a recession affects an MBS is still through mortgage defaults from borrowers being unable to make payments. However, it is also interesting to note that in a deal like this, where some of the mortgages are fixed-rate but all of the securities are floating-rate, signing a swap agreement - meant to provide insurance against not being able to make interest payments to investors -- can backfire if global interest rates are slashed in response to a recession.


\subsection*{To What Extent Were Investors Protected From Mortgage Defaults?}

By summing up the three payment types described in Figure \ref{fig:timeseries_security_principal_pmts_composition} from January 2007 to March 2020, I can now reach a pivotal insight about this deal’s structure, which is that on average, every \$1 of principal received by investors in this deal corresponded to approximately \$0.69 of principal payments from mortgage borrowers, \$0.37 of excess spread from borrowers’ interest payments, and -\$0.06 being sent away to the swap provider. Since it covers the entire life of the deal, not just the time before the overcollateralization buffer was completely emptied, this highlights the importance of the excess spread mechanism for protecting investors, since a significant portion of the principal that investors received actually didn’t come from borrowers’ mortgage principal payments. The total OC buffer in this deal was roughly \$55 million at its inception in January 2007, which on its own would not have been nearly enough to meaningfully protect investors, given the very high volume of defaults throughout the Great Recession.

This “principal decomposition” fits in nicely with another important result regarding this deal’s performance, which is that from January 2007 through March 2020, only 50.6\% of the principal writeoffs on the deal’s underlying mortgages were actually felt by investors. This is calculated by dividing the total principal markdowns for the securities by the total realized losses in the mortgage pool. This still adds up to hundreds of millions of dollars in losses for investors, but there is something to be said for the fact that the structure of this MBS protected its investors from almost half of the defaults on the assets they were exposed to.
	
Of course, the overall “protection rate” of 49.4\% only works at the level of the entire deal, which makes it of limited use for investors, because it’s unlikely that anybody would own stakes in every single tranche. It’s worth remembering that in Group I, all of the mezzanine securities from I-M-3 down to I-M-9 were completely wiped out by writedowns, and Group II did even worse from that standpoint, with all nine II-M mezzanine tranches being written off. At first glance, when considering that more marketed securities were marked down to \$0 than were not, it looks like this deal was almost completely wiped out by the financial crisis, but as described in Part II, the mezzanine securities still made up only a small portion of the deal’s overall principal. Despite all of the public drama surrounding over-zealous speculation on high-yielding mezzanine tranches, they are still only a small proportion of the overall certificate balance of MBS deals like BSABS 2006-HE10 – the bulk of the principal is still held in the senior securities, which will take several more years to be paid off in full and have not suffered such dramatic losses.

The stratified nature of this deal does make it hard to give a single declaration as to whether investors ended up being protected from poor mortgage performance, but the results seem to be less dire than an outsider might expect. The “static” overcollateralization buffer built into each deal provided only brief protection against losses during the early days of the financial crisis, but the “dynamic” mechanism of excess spread turned out to be much more important. The prevailing low-interest-rate environment caused this second source of protection to become the dominant one for investors, because the deal ended up generating enough cash from fixed-rate interest payments to have some left over to mitigate security holders’ losses. 

\subsection*{Maiden Lane: What Happened to the Fed's Money?}

After looking into the extent of the connection between mortgage defaults and security writedowns, I can now finally answer one of the most important questions from earlier, namely whether the Federal Reserve ended up losing any money on the two securities from this deal that it took on during the bailout of Bear Stearns, and more generally, whether those two securities really were so toxic that JPMorgan simply couldn’t bear to take them on. More specifically, did the two tranches of this deal which were shifted onto the Fed’s books during the Bear Stearns bailout cause significant losses for the government? The agreement to salvage what remained of Bear Stearns arose within a number of days against the backdrop of a global financial crisis, which naturally casts some doubt on whether any in-depth analysis of Bear’s MBS holdings was actually done at the time. Now, with the benefit of hindsight, it seems quite valuable to unpack the contributions of our chosen MBS deal to the Maiden Lane fund’s performance, in order to clarify whether the assets that the Fed had to put on its books were really as risky as they were assumed to be at the time. With taxpayers’ money at stake, any significant purchase of securities by the Fed – even if this one was technically a loan made by the New York Fed to the separate Maiden Lane entity – deserves scrutiny, and looking back at the performance of the two MBS tranches discussed here will hopefully improve our understanding of the ramifications of one of the federal government’s most dramatic interventions in financial markets.

	To get started with unpacking the performance of BSABS 2006-HE10’s two Maiden Lane securities, Table \ref{tab:table_maiden_lane_performance} and Figures \ref{fig:timeseries_maiden_lane_ia3} and \ref{fig:timeseries_maiden_lane_ii1a3} provide some summary information about the two tranches’ status at origination and their performance since 2007.

\begin{table}[H]
	\centering
	\begin{tabular}{| p{0.4\linewidth} | p{0.25\linewidth} | p{0.25\linewidth} |}
\hline
 & \textbf{I-A-3} & \textbf{II-1A-3} \\
\hline
Initial Principal Balance & \$11,213,000 & \$20,339,000 \\
\hline
Principal Payments & \$0 & \$0 \\
\hline
Interest Payments & \$2,168,104 & \$3,735,896 \\
\hline
Principal Writedowns & \$0 & \$3,705,585 \\
\hline
Initial Pass-Through Rate & 5.59\% & 5.59\% \\
\hline
Most Recent Pass-Through Rate & 1.87\% & 1.87\% \\
\hline
Initial Credit Rating (Moody’s) & Aaa & Aaa \\
\hline
Most Recent Credit Rating (Moody’s) & Baa3 (lowest investment-grade rating) & Ca (non-investment grade) \\
\hline
Principal “Loss Rate” & 0\% & 18.2\% \\
\hline
Remaining Principal Balance & \$11,213,000 & \$16,633,415 \\
\hline
\end{tabular}

	\caption{Performance Summary for Tranches I-A-3 and II-1A-3 (Data Through March 2020)}
	\label{tab:table_maiden_lane_performance}
\end{table}

\begin{figure}[H]
	\centering
	\caption{Remaining Principal Balance for Tranche I-A-3}
	\includegraphics[width=0.7\textwidth]{../figures/timeseries_maiden_lane_ia3}
	\label{fig:timeseries_maiden_lane_ia3}
	\floatfoot*{Notes: The top panel shows monthly security principal writedowns for Tranche I-A-3, and the bottom shows the tranche's Moody's credit rating. This tranche has not had any of its principal written off since the deal's inception, and its credit rating has recovered after being cut in 2010.}
\end{figure}

\begin{figure}[H]
	\centering
	\caption{Remaining Principal Balance for Tranche II-1A-3}
	\includegraphics[width=0.7\textwidth]{../figures/timeseries_maiden_lane_ii1a3}
	\label{fig:timeseries_maiden_lane_ii1a3}
	\floatfoot*{Notes: This figure contains the same information as Figure \ref{fig:timeseries_maiden_lane_ia3}, except it's for Tranche II-1A-3. This tranche has had about 18\% of its principal written off due to mortgage defaults, and it has had the lowest possible Moody's credit rating since 2010.}
\end{figure}

Neither security has taken particularly heavy losses so far, although II-1A-3 is currently exposed to further writedowns because all of the mezzanine-level tranches in Group II have been fully written off. Both securities have received significant interest payments, although this information is difficult to use, because estimating rates of return on MBS products is a complex and imperfect art. In addition, both tranches started off with perfect Aaa credit ratings, but their ratings have diverged significantly since then: I-A-3 was downgraded dramatically but has recovered its credit rating partially in recent years, while II-1A-3 has not recovered from its series of downgrades during the crisis and is still stuck in junk-bond purgatory today. The divergence in ratings is explained by the fact that I-A-3 still has mezzanine principal below it to absorb any further losses, while II-1A-3 has lost all of its subordination-based credit enhancement and is now directly exposed to future mortgage defaults.

From this brief summary, it looks like at least for this specific deal, the Fed’s purchase of Bear Stearns’s most toxic MBS assets ended up being less dangerous than expected. Since both of these securities are relatively high in the seniority hierarchy, they were well-protected from losses, although II-1A-3 was not quite as bulletproof as its original Aaa rating from 2007 implied. Overall, these securities seem to fit into the “lower senior” group described in Part II, with relatively low risk of losses, but also low returns and a very long actual maturity. Going forward, it seems likely that these two tranches will be some of the last to be paid off in the entire deal, as they each have at least one senior security above them that would receive future principal payments first. In addition, we can also predict relatively low returns for both - and possibly negative returns for II-1A-3 if its writedowns continue - as they started off with relatively low coupon rates in 2007 (compared to the mezzanine tranches), and those pass-through rates have fallen precipitously since then.

In the chaos of a few short weeks when Bear Stearns’s situation went from bad to worse and an urgent rescue was needed, it was easy to lump these two securities in with the pile of “all subprime MBS” and assume that they were going to perform terribly, but with the benefit of over a decade of hindsight, it looks like neither of them was quite as dangerous as JPMorgan or the Fed assumed when the two parties agreed on how to divide up Bear’s assets. This fits in with the general trend of the Maiden Lane fund’s performance, as it only took until June 2012 for the New York Fed’s \$29B loan to Maiden Lane LLC to be fully paid off with interest \parencite{schaefer12}. This allows us to be cautiously optimistic about what would happen if a similar crisis were to occur in the future. Of course, mezzanine tranches like the ones from this deal would be a much riskier bet for the government to take on, but if future debates regarding the insolvency of major financial firms hinge on how to judge securities of roughly this caliber (lower senior), then banks and the government may want to look more closely at whether the securities in question, like I-A-3 and II-1A-3 from this deal, are really so toxic as to truly require being put into a “bad bank” by the Fed. That’s much more easily said than done, but it does offer a small benchmark for evaluating mortgage-backed securities when future financial chaos arises.

\section*{Conclusion}
The previous sections have demonstrated that in this deal, the securities and the underlying mortgages performed very differently: the deal's credit enhancement mechanisms frequently cushioned the blows felt by investors when borrowers defaulted, although mezzanine investors did still suffer significant losses. Beyond this particular deal, this analysis of how its component parts weathered the 2007-08 financial crisis and subsequent slow economic recovery can add to the more general study of securitization and its role in the financial system. Economists have gone to great lengths to characterize the incentive structures that arise when banks securitize their loans, and the theory is valuable for understanding the ways in which stakeholders could potentially mislead others and thus affect the deal’s performance, but it is also helpful to go to the data and understand how well an actual securitization responded to a financial crisis.

Given its large principal base ($>$\$1B) and multiple methods of credit enhancement, BSABS 2006-HE10 looks to be a good test case for studying mortgage-backed securities in general. In a partial vindication of the senior tranches’ original AAA ratings, the top security in Group I has not yet lost any principal; however, the top tranches in Group II have actually suffered some losses, showing that the appraisals of credit risk provided by ratings were not as accurate as investors might have hoped. Investors in mezzanine tranches were more or less cleaned out, as they bet big to take advantage of high potential interest rates - coupons on the most junior tranches were above 7\% at the deal’s inception - but were subsequently burned when macroeconomic conditions worsened. Squeezed in between these high-risk and low-risk segments are securities like I-A-3 and II-1A-3, those that ended up on Maiden Lane’s balance sheet during the dissolution of Bear Stearns, which serve as examples of the “lower senior” segment that offers neither near-guaranteed return of principal nor high potential returns.

\subsection*{MBS-Related Lawsuits Against Bear Stearns}

Although the Fed’s purchase of assorted MBS from Bear Stearns’s ended up yielding profits for the government, the debate over whether these types of securities were appropriate for the general population of investors is made more complicated by the many lawsuits filed against Bear Stearns, which often directly involved this deal. The most important of these suits was filed by the Department of Justice against JPMorgan Chase, which assumed responsibility for MBS products issued by Bear Stearns and Washington Mutual. The DOJ’s investigation uncovered the fact that Bear Stearns executives sometimes deliberately ignored statements from loan originators about how new loans did not meet the originators’ underwriting guidelines, choosing to purchase those loans for MBS deals anyway \parencite{ftc08}. In addition, after disqualifying certain mortgage originators from contributing loans for future securitizations due to poor mortgage performance, Bear Stearns bankers failed to actually re-examine or repurchase those sellers’ low-quality loans which were already present in Bear Stearns-issued MBS, marking another example of how the bank failed to inform investors that the mortgages they were investing in were weaker than what they were promised (DOJ 2013). Ultimately, JPMorgan was forced to admit that it had made “serious misrepresentations” to investors by securitizing loans that did not meet advertised underwriting guidelines, and it was required to pay out a total of \$13 billion. This consisted of a \$2 billion fine for JPMorgan’s own activities, \$4 billion in various forms of debt relief for homeowners whose mortgages had been securitized, and most of the remainder going to the FHFA and other large investors who had bought questionable securities issued by Bear Stearns, Washington Mutual, and JPMorgan itself \parencite{eavis13}. 

In a separate complaint by the SEC against Bear Stearns, it was also revealed that the bank failed to pressure originators to re-purchase bad loans from MBS trusts if the loans missed any payments in the deal’s first three months, which was a provision in many deals’ prospectuses. Instead of requiring the originator to buy back each severely delinquent loan, Bear Stearns would often accept a bulk payment from the cash-strapped originator to address many delinquencies at once, and would then hold that cash on its own books, while leaving those low-quality loans remaining in the mortgage pools of the MBS that it had sold to investors \parencite{sec13}. It's not clear whether this practice specifically occurred for BSABS 2006-HE10, but it’s still highly concerning that this much misconduct was going on in the vicinity of deals such as the one examined here.

On top of the scrutiny it faced from the federal government, JPMorgan ended up having to answer for its own questionable practices, not just the issues it inherited from Bear Stearns. A few years after the crisis, JPMorgan was sued by several large institutional investors, alleging that the bank’s mortgage servicing practices did not live up to the standards laid out in the Pooling and Servicing Agreements that covered many deals originated by Bear Stearns and JPMorgan itself \parencite{gibbs13}. In 2013, JPMorgan agreed to pay \$4.5 billion to various MBS trustees to settle these claims, and it also committed to making several improvements to its servicing practices. BSABS 2006-HE10 acquired another stain on its record by being officially cited in the list of securities covered by the settlement.

\subsection*{What Comes Next for Securitization?}

To finish off a full consideration of this deal’s legacy, it is worth considering how the world of securitization has changed since the Great Recession, and how new regulations might affect future deals like this one. To remedy some of the moral hazard present in the securitization process, US regulations now require the issuer of a securitized product to retain no less than 5\% of its credit risk, either by owning 5\% of each tranche or by owning the first-loss (bottom) tranche as well as enough of the other securities to reach a total of 5\% \parencite{scheicher17}. Other new rules have also introduced more elaborate capital requirements, along with dramatically higher weightings for securitizations in the risk-weighted capital tests that are used to judge banks’ degrees of leverage \parencite{scheicher17}. In reality, these new regulations have so far been applied to a much-diminished market: \textcite{gorton12} point out that the 2007-08 crisis put an almost complete halt to new asset-backed securities issuance. Both sell-side and buy-side arguments could potentially explain this. One the side of MBS issuers, the credit crunch after the 2007 housing crash, and the associated tightening of lending standards, would have dramatically reduced the number of new loans available to be securitized. In addition, it would not be surprising if investors had become much more suspicious of the complex instruments that they had previously been eager to buy, and they may have also been responding to the public stigma surrounding mortgage-backed securities that resulted from these assets’ role in the crisis.

Since the events of 2008, there has been significant research activity aimed at refining our understanding of where the securitization process can go wrong, with a lot of attention on the massive role that credit ratings play in helping investors decide whether to buy any asset-backed security. As just one example, \textcite{efing14} found that credit rating agencies were more likely to give high ratings to deals issued by firms that they did a lot of business with, providing evidence that significant conflicts of interest could lead to the distribution of inflated credit ratings for mediocre securities. This seems like a compelling argument when paired with the fact that every single marketed tranche of BSABS 2006-HE10 received an investment-grade rating at the deal’s inception in 2007, despite the fact that the lower tranches had minimal subordination protection; of course, most of the mezzanine tranches would go on to be completely wiped out thanks to principal writedowns. Results like that of Efing and Hau were behind the post-recession efforts of global standards-setting bodies like the Financial Stability Board to find ways to lessen investors’ dependence on credit ratings for evaluating MBS. Even several years before the financial crisis, it was well known that credit rating agencies were fundamentally tied into the world of securitization: as of 1998, according to Fitch Ratings’s then-vice chairman Neil Baron, “Without ratings, the complex securities in the mortgage field which rely so heavily on the strength of credit enhancements might not be able to be sold” \parencite{baron98}. Even coming from a ratings agency executive who had a lot to gain from making his ratings seem indispensable, it is still concerning to hear that opaque credit ratings play such an instrumental role in the basic process of securitization, and this implies that giving investors ways to avoid fixating on opaque credit ratings when choosing which MBS to invest in will be a great challenge. Some progress has been made, but there is still more that needs to be done to make this aspect of the securitization process less vulnerable to misconduct in the future \parencite{scheicher17}.

In conclusion, this deal offers a wide variety of insights into the practice of securitization at the peak of its popularity. The deal's performance during the financial crisis demonstrates that credit enhancement mechanisms really did protect investors, but also that once defaults begin to accumulate at a sufficient pace, even the best structuring techniques can’t fully prevent losses on an MBS deal’s lower tranches. Because the data I used for this study only cover January 2007 through March 2020, a promising future area of research would be to examine how well similar MBS deals have responded to the COVID-19 pandemic, a broad macroeconomic shock that put new financial pressure on mortgage borrowers nationwide. In theory, these securities will still be making payments until 2036, so much more loan-related and deal-related data will be produced in the coming years, and hopefully this deal will continue to serve as a valuable example for studying how mortgage-backed securities perform in times both good and bad.

\newpage

\nocite{*}
\printbibliography

\end{document}
